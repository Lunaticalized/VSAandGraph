% % Title and author(s)
%%%%%%%%%%%%%%%%%%%%%%%%%%%%%%%%%%%%%%%%%%%%%%%%%%%%%%%
\title{CS 703 Progress Report}
\author{Xiating Ouyang}
\date{}
%%%%%%%%%%%%%%%%%%%%%%%%%%%%%%%%%%%%%%%%%%%%%%%%%%%%%%%
\documentclass{article}
%%%%%%%%%%%%%%%%%%%%%%%%%%%%%%%%%%%%%%%%%%%%%%%%%%%%%%%
% %
% % The next command allows your in import encapsulated
% % postscript files, .epsf or .eps files, which
% % contain vector graphic image data.
% %
%%%%%%%%%%%%%%%%%%%%%%%%%%%%%%%%%%%%%%%%%%%%%%%%%%%%%%%
\usepackage{graphicx}
%\usepackage{charter,eulervm}

%\renewcommand{\baselinestretch}{1.5}

\usepackage{amsthm,amsmath,amssymb,upgreek,marvosym,mathtools}
\usepackage{array}
\usepackage{makeidx}  % allows for indexgeneration
\usepackage{paralist}
\usepackage{subfig}
\usepackage{tabularx}
\usepackage{tabu}
\usepackage[nottoc]{tocbibind}
\usepackage[usenames,dvipsnames]{color}
\usepackage[pdftex,breaklinks,colorlinks,citecolor={blue}, linkcolor={blue},urlcolor=Maroon]{hyperref}
\usepackage{tkz-graph}
\usepackage{geometry}
 \geometry{
 a4paper,
 total={170mm,257mm},
 left=25mm,
 top=20mm,
 right=25mm,
 }
\usetikzlibrary{automata, positioning,arrows,shapes,decorations.pathmorphing}

 \tikzset{
->, % makes the edges directed
>=stealth, % makes the arrow heads bold
node distance=3cm, % specifies the minimum distance between two nodes. Change if necessary.
every state/.style={thick, fill=gray!10}, % sets the properties for each ’state’ node
initial text=$ $, % sets the text that appears on the start arrow
}

\newtheorem{theorem}{Theorem}
\newtheorem{lemma}{Lemma}
\newtheorem{reduction}{Reduction}
\newtheorem{proposition}{Proposition}
\newtheorem{scolium}{Scolium}   %% And a not so common one.
\newtheorem{definition}{Definition}
%\newenvironment{proof}{{\sc Proof:}}{~\hfill QED}
\newenvironment{AMS}{}{}
\newenvironment{keywords}{}{}
\DeclarePairedDelimiter{\norm}{\lVert}{\rVert}
\newcommand{\todo}{(TO BE CONTINUED...)}
\newcommand{\trans}[1]{
	#1^\mathsf{T}
}

\begin{document}
\newpage
\maketitle

\section{Project selection}
The term project is finally settled to be synthesis by examples with a focus on studying the expected size of the maximum clique in a graph following some random distributions. The ultimate goal is to find some probablistic graph models on which the synthesis by examples works practically.

\section{Current progress}
The current research work involves reviewing existing literatures on graph models \cite{aiello2000random,easley2010networks} and the history of this problem \cite{gulwani2012spreadsheet,rolim2017learning,d2017nofaq}. 

From the implementation perspective, a program generating Erd\H{o}s-R\'enyi graph and computing the maximal cliques have been completed for future testing purposes. The implementation of generation of power law graphs is also undergoing [].

 
\bibliography{progress}{}
\bibliographystyle{plain}

\end{document}
